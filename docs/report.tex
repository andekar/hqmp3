\documentclass[a4paper,12pt]{article}
\usepackage[utf8]{inputenc}
\usepackage{graphicx}
\usepackage[superscript]{cite}

\title{HQmp3}
\author{Anders Karlsson \\ \small{\texttt{andekar@student.chalmers.se}}
   \and Tobias Olausson \\ \small{\texttt{olaussot@student.chalmers.se}}
}

\begin{document}
\maketitle

\begin{abstract}
    This project aims to show how one can parse and decode binary formats in
    Haskell, showcased by implementing an mp3 decoder.
\end{abstract}

\tableofcontents

\section{Motivation}
    Det är ju najs med en mp3-decoder i Haskell

\section{Background}
    Hej
    \subsection{The MP3 format}
       The MP3 format is probably the most well-known and widespread format for
       encoding audio. It was engineered by the MPEG group as part of the MPEG-1
       standard ISO-11172 which was published in 1993. As part of this standard,
       the document describing MP3 is ISO-11172-3 \cite{wikimp3,wikimpeg1}.

       MP3 is a lossy format *describe me*.

       The data in an mp3 file is divided into frames. Each frame consists of
       a header, side info, and audio data. *describe me*

\section{Implementation}
    \subsection{bitstring}
        Probably the best library I have ever seen
    \subsection{huffman}
        Probably the most unneccessary library I have ever written. Remove?
        maybe some different approaches should be discussed here? ulf and Koen had their ideas that we discussed and ignored.

\section{Results}
    Beauty, Slowness\ldots

\section{Future Work}

\begin{thebibliography}{99}
    \bibitem{wikimp3}
        Wikipedia on MP3: http://en.wikipedia.org/wiki/Mp3
    \bibitem{wikimpeg1}
        Wikipedia on MPEG-1: http://en.wikipedia.org/wiki/MPEG-1
\end{thebibliography}

\end{document}
